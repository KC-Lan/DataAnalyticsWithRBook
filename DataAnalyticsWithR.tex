\documentclass[]{book}
\usepackage{lmodern}
\usepackage{amssymb,amsmath}
\usepackage{ifxetex,ifluatex}
\usepackage{fixltx2e} % provides \textsubscript
\ifnum 0\ifxetex 1\fi\ifluatex 1\fi=0 % if pdftex
  \usepackage[T1]{fontenc}
  \usepackage[utf8]{inputenc}
\else % if luatex or xelatex
  \ifxetex
    \usepackage{mathspec}
  \else
    \usepackage{fontspec}
  \fi
  \defaultfontfeatures{Ligatures=TeX,Scale=MatchLowercase}
\fi
% use upquote if available, for straight quotes in verbatim environments
\IfFileExists{upquote.sty}{\usepackage{upquote}}{}
% use microtype if available
\IfFileExists{microtype.sty}{%
\usepackage{microtype}
\UseMicrotypeSet[protrusion]{basicmath} % disable protrusion for tt fonts
}{}
\usepackage[margin=1in]{geometry}
\usepackage{hyperref}
\hypersetup{unicode=true,
            pdftitle={資料科學與R語言},
            pdfauthor={曾意儒 Yi-Ju Tseng},
            pdfborder={0 0 0},
            breaklinks=true}
\urlstyle{same}  % don't use monospace font for urls
\usepackage{natbib}
\bibliographystyle{apalike}
\usepackage{color}
\usepackage{fancyvrb}
\newcommand{\VerbBar}{|}
\newcommand{\VERB}{\Verb[commandchars=\\\{\}]}
\DefineVerbatimEnvironment{Highlighting}{Verbatim}{commandchars=\\\{\}}
% Add ',fontsize=\small' for more characters per line
\usepackage{framed}
\definecolor{shadecolor}{RGB}{248,248,248}
\newenvironment{Shaded}{\begin{snugshade}}{\end{snugshade}}
\newcommand{\KeywordTok}[1]{\textcolor[rgb]{0.13,0.29,0.53}{\textbf{{#1}}}}
\newcommand{\DataTypeTok}[1]{\textcolor[rgb]{0.13,0.29,0.53}{{#1}}}
\newcommand{\DecValTok}[1]{\textcolor[rgb]{0.00,0.00,0.81}{{#1}}}
\newcommand{\BaseNTok}[1]{\textcolor[rgb]{0.00,0.00,0.81}{{#1}}}
\newcommand{\FloatTok}[1]{\textcolor[rgb]{0.00,0.00,0.81}{{#1}}}
\newcommand{\ConstantTok}[1]{\textcolor[rgb]{0.00,0.00,0.00}{{#1}}}
\newcommand{\CharTok}[1]{\textcolor[rgb]{0.31,0.60,0.02}{{#1}}}
\newcommand{\SpecialCharTok}[1]{\textcolor[rgb]{0.00,0.00,0.00}{{#1}}}
\newcommand{\StringTok}[1]{\textcolor[rgb]{0.31,0.60,0.02}{{#1}}}
\newcommand{\VerbatimStringTok}[1]{\textcolor[rgb]{0.31,0.60,0.02}{{#1}}}
\newcommand{\SpecialStringTok}[1]{\textcolor[rgb]{0.31,0.60,0.02}{{#1}}}
\newcommand{\ImportTok}[1]{{#1}}
\newcommand{\CommentTok}[1]{\textcolor[rgb]{0.56,0.35,0.01}{\textit{{#1}}}}
\newcommand{\DocumentationTok}[1]{\textcolor[rgb]{0.56,0.35,0.01}{\textbf{\textit{{#1}}}}}
\newcommand{\AnnotationTok}[1]{\textcolor[rgb]{0.56,0.35,0.01}{\textbf{\textit{{#1}}}}}
\newcommand{\CommentVarTok}[1]{\textcolor[rgb]{0.56,0.35,0.01}{\textbf{\textit{{#1}}}}}
\newcommand{\OtherTok}[1]{\textcolor[rgb]{0.56,0.35,0.01}{{#1}}}
\newcommand{\FunctionTok}[1]{\textcolor[rgb]{0.00,0.00,0.00}{{#1}}}
\newcommand{\VariableTok}[1]{\textcolor[rgb]{0.00,0.00,0.00}{{#1}}}
\newcommand{\ControlFlowTok}[1]{\textcolor[rgb]{0.13,0.29,0.53}{\textbf{{#1}}}}
\newcommand{\OperatorTok}[1]{\textcolor[rgb]{0.81,0.36,0.00}{\textbf{{#1}}}}
\newcommand{\BuiltInTok}[1]{{#1}}
\newcommand{\ExtensionTok}[1]{{#1}}
\newcommand{\PreprocessorTok}[1]{\textcolor[rgb]{0.56,0.35,0.01}{\textit{{#1}}}}
\newcommand{\AttributeTok}[1]{\textcolor[rgb]{0.77,0.63,0.00}{{#1}}}
\newcommand{\RegionMarkerTok}[1]{{#1}}
\newcommand{\InformationTok}[1]{\textcolor[rgb]{0.56,0.35,0.01}{\textbf{\textit{{#1}}}}}
\newcommand{\WarningTok}[1]{\textcolor[rgb]{0.56,0.35,0.01}{\textbf{\textit{{#1}}}}}
\newcommand{\AlertTok}[1]{\textcolor[rgb]{0.94,0.16,0.16}{{#1}}}
\newcommand{\ErrorTok}[1]{\textcolor[rgb]{0.64,0.00,0.00}{\textbf{{#1}}}}
\newcommand{\NormalTok}[1]{{#1}}
\usepackage{longtable,booktabs}
\usepackage{graphicx,grffile}
\makeatletter
\def\maxwidth{\ifdim\Gin@nat@width>\linewidth\linewidth\else\Gin@nat@width\fi}
\def\maxheight{\ifdim\Gin@nat@height>\textheight\textheight\else\Gin@nat@height\fi}
\makeatother
% Scale images if necessary, so that they will not overflow the page
% margins by default, and it is still possible to overwrite the defaults
% using explicit options in \includegraphics[width, height, ...]{}
\setkeys{Gin}{width=\maxwidth,height=\maxheight,keepaspectratio}
\IfFileExists{parskip.sty}{%
\usepackage{parskip}
}{% else
\setlength{\parindent}{0pt}
\setlength{\parskip}{6pt plus 2pt minus 1pt}
}
\setlength{\emergencystretch}{3em}  % prevent overfull lines
\providecommand{\tightlist}{%
  \setlength{\itemsep}{0pt}\setlength{\parskip}{0pt}}
\setcounter{secnumdepth}{5}
% Redefines (sub)paragraphs to behave more like sections
\ifx\paragraph\undefined\else
\let\oldparagraph\paragraph
\renewcommand{\paragraph}[1]{\oldparagraph{#1}\mbox{}}
\fi
\ifx\subparagraph\undefined\else
\let\oldsubparagraph\subparagraph
\renewcommand{\subparagraph}[1]{\oldsubparagraph{#1}\mbox{}}
\fi

%%% Use protect on footnotes to avoid problems with footnotes in titles
\let\rmarkdownfootnote\footnote%
\def\footnote{\protect\rmarkdownfootnote}

%%% Change title format to be more compact
\usepackage{titling}

% Create subtitle command for use in maketitle
\newcommand{\subtitle}[1]{
  \posttitle{
    \begin{center}\large#1\end{center}
    }
}

\setlength{\droptitle}{-2em}
  \title{資料科學與R語言}
  \pretitle{\vspace{\droptitle}\centering\huge}
  \posttitle{\par}
  \author{曾意儒 Yi-Ju Tseng}
  \preauthor{\centering\large\emph}
  \postauthor{\par}
  \predate{\centering\large\emph}
  \postdate{\par}
  \date{2017-03-19}

\usepackage{booktabs}
\usepackage{amsthm}

\usepackage{CJKutf8}
\usepackage{xeCJK}  %讓中英文字體分開設置
\setCJKmainfont{微軟正黑體}  
  %設定中文為系統上的字型,而英文不去更動,使用原TeX字型
\XeTeXlinebreaklocale "zh"  
\XeTeXlinebreakskip = 0pt plus 1pt %這兩行一定要加,中文才能自動換行

\makeatletter
\def\thm@space@setup{%
  \thm@preskip=8pt plus 2pt minus 4pt
  \thm@postskip=\thm@preskip
}
\makeatother

\begin{document}
\maketitle

{
\setcounter{tocdepth}{1}
\tableofcontents
}
\chapter*{}\label{preface}
\addcontentsline{toc}{chapter}{}

\chapter{R語言101}\label{intro}

\chapter{R 資料結構}\label{RDataStructure}

\chapter{控制流程}\label{controlstructure}

\chapter{函數}\label{function}

\chapter{資料讀取與匯出}\label{io}

\chapter{資料處理與清洗}\label{manipulation}

\section{Tidy Data}\label{tidy-data}

Each column is a variable. Each row is an observation.

\begin{itemize}
\tightlist
\item
  一個欄位(Column)內只有一個數值,最好要有凡人看得懂的Column Name
\item
  不同的觀察值應該要在不同行(Raw)
\item
  一張表裡面,有所有分析需要的資料
\item
  如果一定要多張表,中間一定要有index可以把表串起來
\item
  One file, one table
\end{itemize}

\section{資料型別轉換處理}

在資料型態章節Chapter \ref{DataType}中,曾介紹\textbf{數值
(numeric)}、\textbf{字串 (character)}、\textbf{布林變數
(logic)}以及\textbf{日期
(Date)}等資料型態,在此章節中將介紹如何檢查變數型別與各型別的轉換。

\subsection{資料型別檢查}

使用以下\texttt{is.}函數檢查資料型別,回傳布林變數,若為真,回傳TRUE

\begin{itemize}
\tightlist
\item
  是否為\textbf{數字} \texttt{is.numeric(變數名稱)}
\item
  是否為\textbf{文字} \texttt{is.character(變數名稱)}
\item
  是否為\textbf{布林變數} \texttt{is.logical(變數名稱)}
\end{itemize}

\begin{Shaded}
\begin{Highlighting}[]
\NormalTok{num<-}\DecValTok{100}
\NormalTok{cha<-}\StringTok{'200'}
\NormalTok{boo<-T}
\KeywordTok{is.numeric}\NormalTok{(num)}
\end{Highlighting}
\end{Shaded}

\begin{verbatim}
## [1] TRUE
\end{verbatim}

\begin{Shaded}
\begin{Highlighting}[]
\KeywordTok{is.numeric}\NormalTok{(cha)}
\end{Highlighting}
\end{Shaded}

\begin{verbatim}
## [1] FALSE
\end{verbatim}

\begin{Shaded}
\begin{Highlighting}[]
\KeywordTok{is.character}\NormalTok{(num)}
\end{Highlighting}
\end{Shaded}

\begin{verbatim}
## [1] FALSE
\end{verbatim}

\begin{Shaded}
\begin{Highlighting}[]
\KeywordTok{is.character}\NormalTok{(cha)}
\end{Highlighting}
\end{Shaded}

\begin{verbatim}
## [1] TRUE
\end{verbatim}

\begin{Shaded}
\begin{Highlighting}[]
\KeywordTok{is.logical}\NormalTok{(boo)}
\end{Highlighting}
\end{Shaded}

\begin{verbatim}
## [1] TRUE
\end{verbatim}

或使用\texttt{class(變數名稱)}函數,直接回傳資料型別

\begin{Shaded}
\begin{Highlighting}[]
\KeywordTok{class}\NormalTok{(num)}
\end{Highlighting}
\end{Shaded}

\begin{verbatim}
## [1] "numeric"
\end{verbatim}

\begin{Shaded}
\begin{Highlighting}[]
\KeywordTok{class}\NormalTok{(cha)}
\end{Highlighting}
\end{Shaded}

\begin{verbatim}
## [1] "character"
\end{verbatim}

\begin{Shaded}
\begin{Highlighting}[]
\KeywordTok{class}\NormalTok{(boo)}
\end{Highlighting}
\end{Shaded}

\begin{verbatim}
## [1] "logical"
\end{verbatim}

\begin{Shaded}
\begin{Highlighting}[]
\KeywordTok{class}\NormalTok{(}\KeywordTok{Sys.Date}\NormalTok{())}
\end{Highlighting}
\end{Shaded}

\begin{verbatim}
## [1] "Date"
\end{verbatim}

\subsection{資料型別轉換}

使用以下\texttt{as.}函數轉換型別

\begin{itemize}
\tightlist
\item
  轉換為\textbf{數字} \texttt{as.numeric(變數名稱)}
\item
  轉換為\textbf{文字} \texttt{as.character(變數名稱)}
\item
  轉換為\textbf{布林變數} \texttt{as.logical(變數名稱)}
\end{itemize}

\begin{Shaded}
\begin{Highlighting}[]
\KeywordTok{as.numeric}\NormalTok{(cha)}
\end{Highlighting}
\end{Shaded}

\begin{verbatim}
## [1] 200
\end{verbatim}

\begin{Shaded}
\begin{Highlighting}[]
\KeywordTok{as.numeric}\NormalTok{(boo)}
\end{Highlighting}
\end{Shaded}

\begin{verbatim}
## [1] 1
\end{verbatim}

\begin{Shaded}
\begin{Highlighting}[]
\KeywordTok{as.character}\NormalTok{(num)}
\end{Highlighting}
\end{Shaded}

\begin{verbatim}
## [1] "100"
\end{verbatim}

\begin{Shaded}
\begin{Highlighting}[]
\KeywordTok{as.character}\NormalTok{(boo)}
\end{Highlighting}
\end{Shaded}

\begin{verbatim}
## [1] "TRUE"
\end{verbatim}

若無法順利完成轉換,會回傳空值\texttt{NA},並出現警告訊息\texttt{Warning:\ NAs\ introduced\ by\ coercion,Warning:\ 強制變更過程中產生了\ NA}

\begin{Shaded}
\begin{Highlighting}[]
\KeywordTok{as.numeric}\NormalTok{(}\StringTok{"abc"}\NormalTok{)}
\end{Highlighting}
\end{Shaded}

\begin{verbatim}
## Warning: NAs introduced by coercion
\end{verbatim}

\begin{verbatim}
## [1] NA
\end{verbatim}

日期的轉換則建議使用\texttt{lubridate}\citep{R-lubridate}
package,如果想要將\texttt{年/月/日}格式的文字轉換為日期物件,可使用\texttt{ymd()}函數(y表年year,m表月month,d表日day),如果想要將\texttt{月/日/年}格式的文字轉換為日期物件,則使用\texttt{mdy()}函數,以此類推。

\begin{Shaded}
\begin{Highlighting}[]
\KeywordTok{library}\NormalTok{(lubridate)}
\KeywordTok{ymd}\NormalTok{(}\StringTok{'2012/3/3'}\NormalTok{)}
\end{Highlighting}
\end{Shaded}

\begin{verbatim}
## [1] "2012-03-03"
\end{verbatim}

\begin{Shaded}
\begin{Highlighting}[]
\KeywordTok{mdy}\NormalTok{(}\StringTok{'3/3/2012'}\NormalTok{)}
\end{Highlighting}
\end{Shaded}

\begin{verbatim}
## [1] "2012-03-03"
\end{verbatim}

\section{文字字串處理}

\subsection{基本處理}

\begin{itemize}
\tightlist
\item
  切割 \texttt{strsplit()}
\item
  子集 \texttt{substr()}
\item
  大小寫轉換 \texttt{toupper()} \texttt{tolower()}
\item
  兩文字連接 \texttt{paste()} \texttt{paste0()}
\item
  文字取代 \texttt{gsub()}
\item
  前後空白去除 \texttt{str\_trim()}
  需安裝\texttt{stringr}\citep{R-stringr} package
\end{itemize}

\begin{Shaded}
\begin{Highlighting}[]
\KeywordTok{strsplit} \NormalTok{(}\StringTok{"Hello World"}\NormalTok{,}\StringTok{" "}\NormalTok{)}
\end{Highlighting}
\end{Shaded}

\begin{verbatim}
## [[1]]
## [1] "Hello" "World"
\end{verbatim}

\begin{Shaded}
\begin{Highlighting}[]
\KeywordTok{toupper}\NormalTok{(}\StringTok{"Hello World"}\NormalTok{)}
\end{Highlighting}
\end{Shaded}

\begin{verbatim}
## [1] "HELLO WORLD"
\end{verbatim}

\begin{Shaded}
\begin{Highlighting}[]
\KeywordTok{tolower}\NormalTok{(}\StringTok{"Hello World"}\NormalTok{)}
\end{Highlighting}
\end{Shaded}

\begin{verbatim}
## [1] "hello world"
\end{verbatim}

\begin{Shaded}
\begin{Highlighting}[]
\KeywordTok{paste}\NormalTok{(}\StringTok{"Hello"}\NormalTok{, }\StringTok{"World"}\NormalTok{, }\DataTypeTok{sep=}\StringTok{''}\NormalTok{)}
\end{Highlighting}
\end{Shaded}

\begin{verbatim}
## [1] "HelloWorld"
\end{verbatim}

\begin{Shaded}
\begin{Highlighting}[]
\KeywordTok{substr}\NormalTok{(}\StringTok{"Hello World"}\NormalTok{, }\DataTypeTok{start=}\DecValTok{2}\NormalTok{,}\DataTypeTok{stop=}\DecValTok{4}\NormalTok{)}
\end{Highlighting}
\end{Shaded}

\begin{verbatim}
## [1] "ell"
\end{verbatim}

\begin{Shaded}
\begin{Highlighting}[]
\KeywordTok{gsub}\NormalTok{(}\StringTok{"o"}\NormalTok{,}\StringTok{"0"}\NormalTok{,}\StringTok{"Hello World"}\NormalTok{)}
\end{Highlighting}
\end{Shaded}

\begin{verbatim}
## [1] "Hell0 W0rld"
\end{verbatim}

\begin{Shaded}
\begin{Highlighting}[]
\KeywordTok{library}\NormalTok{(stringr)}
\KeywordTok{str_trim}\NormalTok{(}\StringTok{" Hello World "}\NormalTok{)}
\end{Highlighting}
\end{Shaded}

\begin{verbatim}
## [1] "Hello World"
\end{verbatim}

\subsection{搜尋字串}

搜尋字串函數通常使用在\textbf{比對文字向量},文字比對\textbf{有分大小寫},依照回傳值的型態不同,有兩種常用函數,\texttt{grep()}與\texttt{grepl()}:

\begin{itemize}
\tightlist
\item
  回傳符合條件之向量位置(index) \texttt{grep(搜尋條件,要搜尋的向量)}
\item
  回傳每個向量是否符合條件(TRUE or FALSE)
  \texttt{grepl(搜尋條件,要搜尋的向量)}
\end{itemize}

\begin{Shaded}
\begin{Highlighting}[]
\KeywordTok{grep}\NormalTok{(}\StringTok{"A"}\NormalTok{,}\KeywordTok{c}\NormalTok{(}\StringTok{"Alex"}\NormalTok{,}\StringTok{"Tom"}\NormalTok{,}\StringTok{"Amy"}\NormalTok{,}\StringTok{"Joy"}\NormalTok{,}\StringTok{"Emma"}\NormalTok{)) ##在姓名文字向量中尋找A,回傳包含"A"之元素位置}
\end{Highlighting}
\end{Shaded}

\begin{verbatim}
## [1] 1 3
\end{verbatim}

\begin{Shaded}
\begin{Highlighting}[]
\KeywordTok{grepl}\NormalTok{(}\StringTok{"A"}\NormalTok{,}\KeywordTok{c}\NormalTok{(}\StringTok{"Alex"}\NormalTok{,}\StringTok{"Tom"}\NormalTok{,}\StringTok{"Amy"}\NormalTok{,}\StringTok{"Joy"}\NormalTok{,}\StringTok{"Emma"}\NormalTok{)) ##在姓名文字向量中尋找A,回傳各元素是否包含"A"}
\end{Highlighting}
\end{Shaded}

\begin{verbatim}
## [1]  TRUE FALSE  TRUE FALSE FALSE
\end{verbatim}

\begin{Shaded}
\begin{Highlighting}[]
\KeywordTok{grepl}\NormalTok{(}\StringTok{"a"}\NormalTok{,}\KeywordTok{c}\NormalTok{(}\StringTok{"Alex"}\NormalTok{,}\StringTok{"Tom"}\NormalTok{,}\StringTok{"Amy"}\NormalTok{,}\StringTok{"Joy"}\NormalTok{,}\StringTok{"Emma"}\NormalTok{)) ##在姓名文字向量中尋找a,回傳各元素是否包含"a"}
\end{Highlighting}
\end{Shaded}

\begin{verbatim}
## [1] FALSE FALSE FALSE FALSE  TRUE
\end{verbatim}

\section{子集Subset}\label{subset}

\subsection{一維資料 (向量)}\label{-}

在向量章節\texttt{\{\#vector\}}有介紹使用\texttt{{[}{]}}取出單一或多個元素的方法

\begin{Shaded}
\begin{Highlighting}[]
\NormalTok{letters ##R語言內建資料之一}
\end{Highlighting}
\end{Shaded}

\begin{verbatim}
##  [1] "a" "b" "c" "d" "e" "f" "g" "h" "i" "j" "k" "l" "m" "n" "o" "p" "q"
## [18] "r" "s" "t" "u" "v" "w" "x" "y" "z"
\end{verbatim}

\begin{Shaded}
\begin{Highlighting}[]
\NormalTok{letters[}\DecValTok{1}\NormalTok{] ##取出letters向量的第一個元素}
\end{Highlighting}
\end{Shaded}

\begin{verbatim}
## [1] "a"
\end{verbatim}

\begin{Shaded}
\begin{Highlighting}[]
\NormalTok{letters[}\DecValTok{1}\NormalTok{:}\DecValTok{10}\NormalTok{] ##取出letters向量的前十個元素}
\end{Highlighting}
\end{Shaded}

\begin{verbatim}
##  [1] "a" "b" "c" "d" "e" "f" "g" "h" "i" "j"
\end{verbatim}

\begin{Shaded}
\begin{Highlighting}[]
\NormalTok{letters[}\KeywordTok{c}\NormalTok{(}\DecValTok{1}\NormalTok{,}\DecValTok{3}\NormalTok{,}\DecValTok{5}\NormalTok{)] ##取出letters向量的第1,3,5個元素}
\end{Highlighting}
\end{Shaded}

\begin{verbatim}
## [1] "a" "c" "e"
\end{verbatim}

\begin{Shaded}
\begin{Highlighting}[]
\NormalTok{letters[}\KeywordTok{c}\NormalTok{(-}\DecValTok{1}\NormalTok{,-}\DecValTok{3}\NormalTok{,-}\DecValTok{5}\NormalTok{)] ##取出letters向量除了第1,3,5個元素之外的所有元素}
\end{Highlighting}
\end{Shaded}

\begin{verbatim}
##  [1] "b" "d" "f" "g" "h" "i" "j" "k" "l" "m" "n" "o" "p" "q" "r" "s" "t"
## [18] "u" "v" "w" "x" "y" "z"
\end{verbatim}

若想要快速取得一向量的開頭與結尾元素,可使用\texttt{head()}和\texttt{tail()}函數

\begin{Shaded}
\begin{Highlighting}[]
\KeywordTok{head}\NormalTok{(letters,}\DecValTok{5}\NormalTok{) ##取出letters向量的前五個元素}
\end{Highlighting}
\end{Shaded}

\begin{verbatim}
## [1] "a" "b" "c" "d" "e"
\end{verbatim}

\begin{Shaded}
\begin{Highlighting}[]
\KeywordTok{tail}\NormalTok{(letters,}\DecValTok{3}\NormalTok{) ##取出letters向量的後三個元素}
\end{Highlighting}
\end{Shaded}

\begin{verbatim}
## [1] "x" "y" "z"
\end{verbatim}

\subsection{二維資料}

最常見的二維資料為data.frame資料框,二維資料可針對列(Row)和行(Column)做子集,子集選擇方式一樣是使用\texttt{{[}{]}},但因應二維資料的需求,以\texttt{,}分隔列與行的篩選條件,資料篩選原則為\textbf{前Row,後Column},\textbf{前列,後行},若不想篩選列,則在\texttt{,}前方保持\textbf{空白}即可。

篩選方式可輸入位置(index)、欄位名稱或輸入布林變數(TRUE/FALSE)

\begin{itemize}
\tightlist
\item
  輸入位置: \texttt{dataFrame{[}row\ index,column\ index{]}}
\item
  輸入布林變數: \texttt{dataFrame{[}c(T,F,T),c(T,F,T){]}}
\item
  輸入欄位名稱: \texttt{dataFrame{[}row\ name,column\ name{]}}
\end{itemize}

\begin{Shaded}
\begin{Highlighting}[]
\NormalTok{iris[}\DecValTok{1}\NormalTok{,}\DecValTok{2}\NormalTok{] ##第一列Row,第二行Column}
\end{Highlighting}
\end{Shaded}

\begin{verbatim}
## [1] 3.5
\end{verbatim}

\begin{Shaded}
\begin{Highlighting}[]
\NormalTok{iris[}\DecValTok{1}\NormalTok{:}\DecValTok{3}\NormalTok{,] ##第1~3列Row,所有的行Column}
\end{Highlighting}
\end{Shaded}

\begin{verbatim}
##   Sepal.Length Sepal.Width Petal.Length Petal.Width Species
## 1          5.1         3.5          1.4         0.2  setosa
## 2          4.9         3.0          1.4         0.2  setosa
## 3          4.7         3.2          1.3         0.2  setosa
\end{verbatim}

\begin{Shaded}
\begin{Highlighting}[]
\CommentTok{#iris[,"Species"] ##所有的列Row,名稱為Species的行Column}
\NormalTok{iris[}\DecValTok{1}\NormalTok{:}\DecValTok{10}\NormalTok{,}\KeywordTok{c}\NormalTok{(T,F,T,F,T)] ##第1~10列Row,第1,3,5行Column (TRUE)}
\end{Highlighting}
\end{Shaded}

\begin{verbatim}
##    Sepal.Length Petal.Length Species
## 1           5.1          1.4  setosa
## 2           4.9          1.4  setosa
## 3           4.7          1.3  setosa
## 4           4.6          1.5  setosa
## 5           5.0          1.4  setosa
## 6           5.4          1.7  setosa
## 7           4.6          1.4  setosa
## 8           5.0          1.5  setosa
## 9           4.4          1.4  setosa
## 10          4.9          1.5  setosa
\end{verbatim}

也可使用\texttt{\$}符號做\textbf{Column的篩選}

\begin{Shaded}
\begin{Highlighting}[]
\NormalTok{iris$Species ##所有的列Row,名稱為Species的行Column}
\end{Highlighting}
\end{Shaded}

\begin{verbatim}
##   [1] setosa     setosa     setosa     setosa     setosa     setosa    
##   [7] setosa     setosa     setosa     setosa     setosa     setosa    
##  [13] setosa     setosa     setosa     setosa     setosa     setosa    
##  [19] setosa     setosa     setosa     setosa     setosa     setosa    
##  [25] setosa     setosa     setosa     setosa     setosa     setosa    
##  [31] setosa     setosa     setosa     setosa     setosa     setosa    
##  [37] setosa     setosa     setosa     setosa     setosa     setosa    
##  [43] setosa     setosa     setosa     setosa     setosa     setosa    
##  [49] setosa     setosa     versicolor versicolor versicolor versicolor
##  [55] versicolor versicolor versicolor versicolor versicolor versicolor
##  [61] versicolor versicolor versicolor versicolor versicolor versicolor
##  [67] versicolor versicolor versicolor versicolor versicolor versicolor
##  [73] versicolor versicolor versicolor versicolor versicolor versicolor
##  [79] versicolor versicolor versicolor versicolor versicolor versicolor
##  [85] versicolor versicolor versicolor versicolor versicolor versicolor
##  [91] versicolor versicolor versicolor versicolor versicolor versicolor
##  [97] versicolor versicolor versicolor versicolor virginica  virginica 
## [103] virginica  virginica  virginica  virginica  virginica  virginica 
## [109] virginica  virginica  virginica  virginica  virginica  virginica 
## [115] virginica  virginica  virginica  virginica  virginica  virginica 
## [121] virginica  virginica  virginica  virginica  virginica  virginica 
## [127] virginica  virginica  virginica  virginica  virginica  virginica 
## [133] virginica  virginica  virginica  virginica  virginica  virginica 
## [139] virginica  virginica  virginica  virginica  virginica  virginica 
## [145] virginica  virginica  virginica  virginica  virginica  virginica 
## Levels: setosa versicolor virginica
\end{verbatim}

\textbf{Row的篩選}可使用\texttt{subset()}函數,使用方法為\texttt{subset(資料表,篩選邏輯)}

\begin{Shaded}
\begin{Highlighting}[]
\KeywordTok{subset}\NormalTok{(iris,Species==}\StringTok{"virginica"}\NormalTok{) ##Species等於"virginica"的列Row,所有的行Column}
\end{Highlighting}
\end{Shaded}

\begin{verbatim}
##     Sepal.Length Sepal.Width Petal.Length Petal.Width   Species
## 101          6.3         3.3          6.0         2.5 virginica
## 102          5.8         2.7          5.1         1.9 virginica
## 103          7.1         3.0          5.9         2.1 virginica
## 104          6.3         2.9          5.6         1.8 virginica
## 105          6.5         3.0          5.8         2.2 virginica
## 106          7.6         3.0          6.6         2.1 virginica
## 107          4.9         2.5          4.5         1.7 virginica
## 108          7.3         2.9          6.3         1.8 virginica
## 109          6.7         2.5          5.8         1.8 virginica
## 110          7.2         3.6          6.1         2.5 virginica
## 111          6.5         3.2          5.1         2.0 virginica
## 112          6.4         2.7          5.3         1.9 virginica
## 113          6.8         3.0          5.5         2.1 virginica
## 114          5.7         2.5          5.0         2.0 virginica
## 115          5.8         2.8          5.1         2.4 virginica
## 116          6.4         3.2          5.3         2.3 virginica
## 117          6.5         3.0          5.5         1.8 virginica
## 118          7.7         3.8          6.7         2.2 virginica
## 119          7.7         2.6          6.9         2.3 virginica
## 120          6.0         2.2          5.0         1.5 virginica
## 121          6.9         3.2          5.7         2.3 virginica
## 122          5.6         2.8          4.9         2.0 virginica
## 123          7.7         2.8          6.7         2.0 virginica
## 124          6.3         2.7          4.9         1.8 virginica
## 125          6.7         3.3          5.7         2.1 virginica
## 126          7.2         3.2          6.0         1.8 virginica
## 127          6.2         2.8          4.8         1.8 virginica
## 128          6.1         3.0          4.9         1.8 virginica
## 129          6.4         2.8          5.6         2.1 virginica
## 130          7.2         3.0          5.8         1.6 virginica
## 131          7.4         2.8          6.1         1.9 virginica
## 132          7.9         3.8          6.4         2.0 virginica
## 133          6.4         2.8          5.6         2.2 virginica
## 134          6.3         2.8          5.1         1.5 virginica
## 135          6.1         2.6          5.6         1.4 virginica
## 136          7.7         3.0          6.1         2.3 virginica
## 137          6.3         3.4          5.6         2.4 virginica
## 138          6.4         3.1          5.5         1.8 virginica
## 139          6.0         3.0          4.8         1.8 virginica
## 140          6.9         3.1          5.4         2.1 virginica
## 141          6.7         3.1          5.6         2.4 virginica
## 142          6.9         3.1          5.1         2.3 virginica
## 143          5.8         2.7          5.1         1.9 virginica
## 144          6.8         3.2          5.9         2.3 virginica
## 145          6.7         3.3          5.7         2.5 virginica
## 146          6.7         3.0          5.2         2.3 virginica
## 147          6.3         2.5          5.0         1.9 virginica
## 148          6.5         3.0          5.2         2.0 virginica
## 149          6.2         3.4          5.4         2.3 virginica
## 150          5.9         3.0          5.1         1.8 virginica
\end{verbatim}

\textbf{Row的篩選}也可搭配字串搜尋函數\texttt{grepl()}

\begin{Shaded}
\begin{Highlighting}[]
\NormalTok{knitr::}\KeywordTok{kable}\NormalTok{(iris[}\KeywordTok{grepl}\NormalTok{(}\StringTok{"color"}\NormalTok{,iris$Species),]) ##Species包含"color"的列,所有的行}
\end{Highlighting}
\end{Shaded}

\begin{tabular}{l|r|r|r|r|l}
\hline
  & Sepal.Length & Sepal.Width & Petal.Length & Petal.Width & Species\\
\hline
51 & 7.0 & 3.2 & 4.7 & 1.4 & versicolor\\
\hline
52 & 6.4 & 3.2 & 4.5 & 1.5 & versicolor\\
\hline
53 & 6.9 & 3.1 & 4.9 & 1.5 & versicolor\\
\hline
54 & 5.5 & 2.3 & 4.0 & 1.3 & versicolor\\
\hline
55 & 6.5 & 2.8 & 4.6 & 1.5 & versicolor\\
\hline
56 & 5.7 & 2.8 & 4.5 & 1.3 & versicolor\\
\hline
57 & 6.3 & 3.3 & 4.7 & 1.6 & versicolor\\
\hline
58 & 4.9 & 2.4 & 3.3 & 1.0 & versicolor\\
\hline
59 & 6.6 & 2.9 & 4.6 & 1.3 & versicolor\\
\hline
60 & 5.2 & 2.7 & 3.9 & 1.4 & versicolor\\
\hline
61 & 5.0 & 2.0 & 3.5 & 1.0 & versicolor\\
\hline
62 & 5.9 & 3.0 & 4.2 & 1.5 & versicolor\\
\hline
63 & 6.0 & 2.2 & 4.0 & 1.0 & versicolor\\
\hline
64 & 6.1 & 2.9 & 4.7 & 1.4 & versicolor\\
\hline
65 & 5.6 & 2.9 & 3.6 & 1.3 & versicolor\\
\hline
66 & 6.7 & 3.1 & 4.4 & 1.4 & versicolor\\
\hline
67 & 5.6 & 3.0 & 4.5 & 1.5 & versicolor\\
\hline
68 & 5.8 & 2.7 & 4.1 & 1.0 & versicolor\\
\hline
69 & 6.2 & 2.2 & 4.5 & 1.5 & versicolor\\
\hline
70 & 5.6 & 2.5 & 3.9 & 1.1 & versicolor\\
\hline
71 & 5.9 & 3.2 & 4.8 & 1.8 & versicolor\\
\hline
72 & 6.1 & 2.8 & 4.0 & 1.3 & versicolor\\
\hline
73 & 6.3 & 2.5 & 4.9 & 1.5 & versicolor\\
\hline
74 & 6.1 & 2.8 & 4.7 & 1.2 & versicolor\\
\hline
75 & 6.4 & 2.9 & 4.3 & 1.3 & versicolor\\
\hline
76 & 6.6 & 3.0 & 4.4 & 1.4 & versicolor\\
\hline
77 & 6.8 & 2.8 & 4.8 & 1.4 & versicolor\\
\hline
78 & 6.7 & 3.0 & 5.0 & 1.7 & versicolor\\
\hline
79 & 6.0 & 2.9 & 4.5 & 1.5 & versicolor\\
\hline
80 & 5.7 & 2.6 & 3.5 & 1.0 & versicolor\\
\hline
81 & 5.5 & 2.4 & 3.8 & 1.1 & versicolor\\
\hline
82 & 5.5 & 2.4 & 3.7 & 1.0 & versicolor\\
\hline
83 & 5.8 & 2.7 & 3.9 & 1.2 & versicolor\\
\hline
84 & 6.0 & 2.7 & 5.1 & 1.6 & versicolor\\
\hline
85 & 5.4 & 3.0 & 4.5 & 1.5 & versicolor\\
\hline
86 & 6.0 & 3.4 & 4.5 & 1.6 & versicolor\\
\hline
87 & 6.7 & 3.1 & 4.7 & 1.5 & versicolor\\
\hline
88 & 6.3 & 2.3 & 4.4 & 1.3 & versicolor\\
\hline
89 & 5.6 & 3.0 & 4.1 & 1.3 & versicolor\\
\hline
90 & 5.5 & 2.5 & 4.0 & 1.3 & versicolor\\
\hline
91 & 5.5 & 2.6 & 4.4 & 1.2 & versicolor\\
\hline
92 & 6.1 & 3.0 & 4.6 & 1.4 & versicolor\\
\hline
93 & 5.8 & 2.6 & 4.0 & 1.2 & versicolor\\
\hline
94 & 5.0 & 2.3 & 3.3 & 1.0 & versicolor\\
\hline
95 & 5.6 & 2.7 & 4.2 & 1.3 & versicolor\\
\hline
96 & 5.7 & 3.0 & 4.2 & 1.2 & versicolor\\
\hline
97 & 5.7 & 2.9 & 4.2 & 1.3 & versicolor\\
\hline
98 & 6.2 & 2.9 & 4.3 & 1.3 & versicolor\\
\hline
99 & 5.1 & 2.5 & 3.0 & 1.1 & versicolor\\
\hline
100 & 5.7 & 2.8 & 4.1 & 1.3 & versicolor\\
\hline
\end{tabular}

若想要快速取得資料框的前幾列(Raw)或後幾列,也可使用\texttt{head()}和\texttt{tail()}函數

\begin{Shaded}
\begin{Highlighting}[]
\KeywordTok{head}\NormalTok{(iris,}\DecValTok{5}\NormalTok{) ##取出iris資料框的前五列}
\end{Highlighting}
\end{Shaded}

\begin{verbatim}
##   Sepal.Length Sepal.Width Petal.Length Petal.Width Species
## 1          5.1         3.5          1.4         0.2  setosa
## 2          4.9         3.0          1.4         0.2  setosa
## 3          4.7         3.2          1.3         0.2  setosa
## 4          4.6         3.1          1.5         0.2  setosa
## 5          5.0         3.6          1.4         0.2  setosa
\end{verbatim}

\begin{Shaded}
\begin{Highlighting}[]
\KeywordTok{tail}\NormalTok{(iris,}\DecValTok{3}\NormalTok{) ##取出iris資料框的後三列}
\end{Highlighting}
\end{Shaded}

\begin{verbatim}
##     Sepal.Length Sepal.Width Petal.Length Petal.Width   Species
## 148          6.5         3.0          5.2         2.0 virginica
## 149          6.2         3.4          5.4         2.3 virginica
## 150          5.9         3.0          5.1         1.8 virginica
\end{verbatim}

\section{排序}

\subsection{sort 向量排序}\label{sort-}

\texttt{sort()}函數可直接對向量做\textbf{由小到大}的排序

\begin{Shaded}
\begin{Highlighting}[]
\KeywordTok{head}\NormalTok{(islands) ##排序前的前六筆資料}
\end{Highlighting}
\end{Shaded}

\begin{verbatim}
##       Africa   Antarctica         Asia    Australia Axel Heiberg 
##        11506         5500        16988         2968           16 
##       Baffin 
##          184
\end{verbatim}

\begin{Shaded}
\begin{Highlighting}[]
\KeywordTok{head}\NormalTok{(}\KeywordTok{sort}\NormalTok{(islands)) ##由小到大排序後的前六筆資料}
\end{Highlighting}
\end{Shaded}

\begin{verbatim}
##       Vancouver          Hainan Prince of Wales           Timor 
##              12              13              13              13 
##          Kyushu          Taiwan 
##              14              14
\end{verbatim}

如需\textbf{由大到小}排序,可將\texttt{decreasing}參數設為TRUE

\begin{Shaded}
\begin{Highlighting}[]
\KeywordTok{head}\NormalTok{(}\KeywordTok{sort}\NormalTok{(islands,}\DataTypeTok{decreasing =} \NormalTok{T)) ##由大到小排序後的前六筆資料}
\end{Highlighting}
\end{Shaded}

\begin{verbatim}
##          Asia        Africa North America South America    Antarctica 
##         16988         11506          9390          6795          5500 
##        Europe 
##          3745
\end{verbatim}

\subsection{order}\label{order}

如需對資料框做排序,可使用\texttt{order()}函數,\texttt{order()}函數可回傳\textbf{由小到大}之\textbf{元素位置},以\texttt{iris\$Sepal.Length}為例,回傳的第一個位置為\texttt{14},表示\texttt{iris\$Sepal.Length}中,數值最小的元素為第14個元素。

\begin{Shaded}
\begin{Highlighting}[]
\KeywordTok{order}\NormalTok{(iris$Sepal.Length)}
\end{Highlighting}
\end{Shaded}

\begin{verbatim}
##   [1]  14   9  39  43  42   4   7  23  48   3  30  12  13  25  31  46   2
##  [18]  10  35  38  58 107   5   8  26  27  36  41  44  50  61  94   1  18
##  [35]  20  22  24  40  45  47  99  28  29  33  60  49   6  11  17  21  32
##  [52]  85  34  37  54  81  82  90  91  65  67  70  89  95 122  16  19  56
##  [69]  80  96  97 100 114  15  68  83  93 102 115 143  62  71 150  63  79
##  [86]  84  86 120 139  64  72  74  92 128 135  69  98 127 149  57  73  88
## [103] 101 104 124 134 137 147  52  75 112 116 129 133 138  55 105 111 117
## [120] 148  59  76  66  78  87 109 125 141 145 146  77 113 144  53 121 140
## [137] 142  51 103 110 126 130 108 131 106 118 119 123 136 132
\end{verbatim}

\begin{Shaded}
\begin{Highlighting}[]
\NormalTok{iris$Sepal.Length[}\DecValTok{14}\NormalTok{]}
\end{Highlighting}
\end{Shaded}

\begin{verbatim}
## [1] 4.3
\end{verbatim}

若將\texttt{decreasing}參數設定為TRUE,則會回傳\textbf{由大到小}的元素位置,以\texttt{iris\$Sepal.Length}為例,回傳的第一個位置為\texttt{132},表示\texttt{iris\$Sepal.Length}中,數值最大的元素為第132個元素。

\begin{Shaded}
\begin{Highlighting}[]
\KeywordTok{order}\NormalTok{(iris$Sepal.Length,}\DataTypeTok{decreasing =} \NormalTok{T)}
\end{Highlighting}
\end{Shaded}

\begin{verbatim}
##   [1] 132 118 119 123 136 106 131 108 110 126 130 103  51  53 121 140 142
##  [18]  77 113 144  66  78  87 109 125 141 145 146  59  76  55 105 111 117
##  [35] 148  52  75 112 116 129 133 138  57  73  88 101 104 124 134 137 147
##  [52]  69  98 127 149  64  72  74  92 128 135  63  79  84  86 120 139  62
##  [69]  71 150  15  68  83  93 102 115 143  16  19  56  80  96  97 100 114
##  [86]  65  67  70  89  95 122  34  37  54  81  82  90  91   6  11  17  21
## [103]  32  85  49  28  29  33  60   1  18  20  22  24  40  45  47  99   5
## [120]   8  26  27  36  41  44  50  61  94   2  10  35  38  58 107  12  13
## [137]  25  31  46   3  30   4   7  23  48  42   9  39  43  14
\end{verbatim}

\begin{Shaded}
\begin{Highlighting}[]
\NormalTok{iris$Sepal.Length[}\DecValTok{132}\NormalTok{]}
\end{Highlighting}
\end{Shaded}

\begin{verbatim}
## [1] 7.9
\end{verbatim}

依照order回傳的元素位置,重新排序iris資料框

\begin{Shaded}
\begin{Highlighting}[]
\KeywordTok{head}\NormalTok{(iris) ##排序前的前六筆資料}
\end{Highlighting}
\end{Shaded}

\begin{verbatim}
##   Sepal.Length Sepal.Width Petal.Length Petal.Width Species
## 1          5.1         3.5          1.4         0.2  setosa
## 2          4.9         3.0          1.4         0.2  setosa
## 3          4.7         3.2          1.3         0.2  setosa
## 4          4.6         3.1          1.5         0.2  setosa
## 5          5.0         3.6          1.4         0.2  setosa
## 6          5.4         3.9          1.7         0.4  setosa
\end{verbatim}

\begin{Shaded}
\begin{Highlighting}[]
\KeywordTok{head}\NormalTok{(iris[}\KeywordTok{order}\NormalTok{(iris$Sepal.Length),]) ##依照Sepal.Length欄位數值大小排序後的前六筆資料}
\end{Highlighting}
\end{Shaded}

\begin{verbatim}
##    Sepal.Length Sepal.Width Petal.Length Petal.Width Species
## 14          4.3         3.0          1.1         0.1  setosa
## 9           4.4         2.9          1.4         0.2  setosa
## 39          4.4         3.0          1.3         0.2  setosa
## 43          4.4         3.2          1.3         0.2  setosa
## 42          4.5         2.3          1.3         0.3  setosa
## 4           4.6         3.1          1.5         0.2  setosa
\end{verbatim}

\begin{Shaded}
\begin{Highlighting}[]
\KeywordTok{head}\NormalTok{(iris[}\KeywordTok{order}\NormalTok{(iris$Sepal.Length,}\DataTypeTok{decreasing =} \NormalTok{T),]) ##改為由大到小排序的前六筆資料}
\end{Highlighting}
\end{Shaded}

\begin{verbatim}
##     Sepal.Length Sepal.Width Petal.Length Petal.Width   Species
## 132          7.9         3.8          6.4         2.0 virginica
## 118          7.7         3.8          6.7         2.2 virginica
## 119          7.7         2.6          6.9         2.3 virginica
## 123          7.7         2.8          6.7         2.0 virginica
## 136          7.7         3.0          6.1         2.3 virginica
## 106          7.6         3.0          6.6         2.1 virginica
\end{verbatim}

\section{資料組合}

有時需要在資料框新增一列,或新增一行,可以利用資料組合函數完成

\begin{itemize}
\tightlist
\item
  Row 列的組合 \texttt{rbind()}
\item
  Column 行的組合 \texttt{cbind()}
\end{itemize}

\texttt{rbind()}和\texttt{cbind()}的參數可以是向量,也可以是資料框,使用向量做資料整合範例:

\begin{Shaded}
\begin{Highlighting}[]
\KeywordTok{rbind}\NormalTok{(}\KeywordTok{c}\NormalTok{(}\DecValTok{1}\NormalTok{,}\DecValTok{2}\NormalTok{,}\DecValTok{3}\NormalTok{), }\CommentTok{#第一列}
      \KeywordTok{c}\NormalTok{(}\DecValTok{4}\NormalTok{,}\DecValTok{5}\NormalTok{,}\DecValTok{6}\NormalTok{)  }\CommentTok{#第二列}
      \NormalTok{) }
\end{Highlighting}
\end{Shaded}

\begin{verbatim}
##      [,1] [,2] [,3]
## [1,]    1    2    3
## [2,]    4    5    6
\end{verbatim}

使用資料框與向量做資料整合範例:

\begin{Shaded}
\begin{Highlighting}[]
\NormalTok{irisAdd<-}\KeywordTok{rbind}\NormalTok{(iris, }\CommentTok{#資料框}
      \KeywordTok{c}\NormalTok{(}\DecValTok{1}\NormalTok{,}\DecValTok{1}\NormalTok{,}\DecValTok{1}\NormalTok{,}\DecValTok{1}\NormalTok{,}\StringTok{"versicolor"}\NormalTok{)  }\CommentTok{#新增一列}
      \NormalTok{) }
\KeywordTok{tail}\NormalTok{(irisAdd)}
\end{Highlighting}
\end{Shaded}

\begin{verbatim}
##     Sepal.Length Sepal.Width Petal.Length Petal.Width    Species
## 146          6.7           3          5.2         2.3  virginica
## 147          6.3         2.5            5         1.9  virginica
## 148          6.5           3          5.2           2  virginica
## 149          6.2         3.4          5.4         2.3  virginica
## 150          5.9           3          5.1         1.8  virginica
## 151            1           1            1           1 versicolor
\end{verbatim}

使用向量做資料整合範例:

\begin{Shaded}
\begin{Highlighting}[]
\KeywordTok{cbind}\NormalTok{(}\KeywordTok{c}\NormalTok{(}\DecValTok{1}\NormalTok{,}\DecValTok{2}\NormalTok{,}\DecValTok{3}\NormalTok{), }\CommentTok{#第一行}
      \KeywordTok{c}\NormalTok{(}\DecValTok{4}\NormalTok{,}\DecValTok{5}\NormalTok{,}\DecValTok{6}\NormalTok{)  }\CommentTok{#第二行}
      \NormalTok{) }
\end{Highlighting}
\end{Shaded}

\begin{verbatim}
##      [,1] [,2]
## [1,]    1    4
## [2,]    2    5
## [3,]    3    6
\end{verbatim}

使用資料框與向量做資料整合範例:

\begin{Shaded}
\begin{Highlighting}[]
\NormalTok{irisAdd<-}\KeywordTok{cbind}\NormalTok{(iris, }\CommentTok{#資料框}
      \KeywordTok{rep}\NormalTok{(}\StringTok{"Add"}\NormalTok{,}\KeywordTok{nrow}\NormalTok{(iris))  }\CommentTok{#新增一行}
      \NormalTok{) }
\KeywordTok{tail}\NormalTok{(irisAdd)}
\end{Highlighting}
\end{Shaded}

\begin{verbatim}
##     Sepal.Length Sepal.Width Petal.Length Petal.Width   Species
## 145          6.7         3.3          5.7         2.5 virginica
## 146          6.7         3.0          5.2         2.3 virginica
## 147          6.3         2.5          5.0         1.9 virginica
## 148          6.5         3.0          5.2         2.0 virginica
## 149          6.2         3.4          5.4         2.3 virginica
## 150          5.9         3.0          5.1         1.8 virginica
##     rep("Add", nrow(iris))
## 145                    Add
## 146                    Add
## 147                    Add
## 148                    Add
## 149                    Add
## 150                    Add
\end{verbatim}

\section{長表與寬表}\label{reshape}

在資料處理的過程中,常因各種需求,需要執行長寬表互換的動作,在R中有很好用的套件reshape2\citep{R-reshape2}
package,提供完整的轉換功能,最常使用的是

\begin{itemize}
\tightlist
\item
  寬表轉長表 \texttt{melt(資料框/寬表,id.vars=需要保留的欄位)}
\item
  長表轉寬表
  \texttt{dcast(資料框/長表,寬表分列依據\textasciitilde{}分欄位依據)}
\end{itemize}

原來的\texttt{airquality}資料框中,有Ozone, Solar.R, Wind, Temp, Month,
Day等六個欄位
(Column),屬於寬表,以下範例將保留Month和Day兩個欄位,並將其他欄位的名稱整合至variable欄位,數值整合至value欄位,寬表轉長表範例如下:

\begin{Shaded}
\begin{Highlighting}[]
\KeywordTok{library}\NormalTok{(reshape2)}
\KeywordTok{head}\NormalTok{(airquality)}
\end{Highlighting}
\end{Shaded}

\begin{verbatim}
##   Ozone Solar.R Wind Temp Month Day
## 1    41     190  7.4   67     5   1
## 2    36     118  8.0   72     5   2
## 3    12     149 12.6   74     5   3
## 4    18     313 11.5   62     5   4
## 5    NA      NA 14.3   56     5   5
## 6    28      NA 14.9   66     5   6
\end{verbatim}

\begin{Shaded}
\begin{Highlighting}[]
\NormalTok{airqualityM<-}\KeywordTok{melt}\NormalTok{(airquality,}\DataTypeTok{id.vars =} \KeywordTok{c}\NormalTok{(}\StringTok{"Month"}\NormalTok{,}\StringTok{"Day"}\NormalTok{)) ##欄位需要保留"Month","Day"}
\KeywordTok{head}\NormalTok{(airqualityM)}
\end{Highlighting}
\end{Shaded}

\begin{verbatim}
##   Month Day variable value
## 1     5   1    Ozone    41
## 2     5   2    Ozone    36
## 3     5   3    Ozone    12
## 4     5   4    Ozone    18
## 5     5   5    Ozone    NA
## 6     5   6    Ozone    28
\end{verbatim}

轉換過的長表\texttt{airqualityM}資料框中,剩下Month, Day, variable,
value等四個欄位
(Column),屬於長表,以下範例variable欄位的值轉換為新欄位,並將value欄位填回新增的欄位,長表轉寬表範例如下:

\begin{Shaded}
\begin{Highlighting}[]
\KeywordTok{library}\NormalTok{(reshape2)}
\NormalTok{##欄位保留"Month","Day"外,其他欄位數目由variable定義}
\NormalTok{airqualityCast<-}\KeywordTok{dcast}\NormalTok{(airqualityM, Month +Day~variable) }
\KeywordTok{head}\NormalTok{(airqualityCast)}
\end{Highlighting}
\end{Shaded}

\begin{verbatim}
##   Month Day Ozone Solar.R Wind Temp
## 1     5   1    41     190  7.4   67
## 2     5   2    36     118  8.0   72
## 3     5   3    12     149 12.6   74
## 4     5   4    18     313 11.5   62
## 5     5   5    NA      NA 14.3   56
## 6     5   6    28      NA 14.9   66
\end{verbatim}

\section{遺漏值處理}

遺漏值(Missing
Value)常常出現在真實資料內,在數值運算時常會有問題,最簡單的方法是將有缺值的資料移除,如資料為向量,可使用\texttt{is.na()}來判斷資料是否為空值\texttt{NA},若為真\texttt{TRUE},則將資料移除。

\begin{Shaded}
\begin{Highlighting}[]
\NormalTok{naVec<-}\KeywordTok{c}\NormalTok{(}\StringTok{"a"}\NormalTok{,}\StringTok{"b"}\NormalTok{,}\OtherTok{NA}\NormalTok{,}\StringTok{"d"}\NormalTok{,}\StringTok{"e"}\NormalTok{)}
\KeywordTok{is.na}\NormalTok{(naVec)}
\end{Highlighting}
\end{Shaded}

\begin{verbatim}
## [1] FALSE FALSE  TRUE FALSE FALSE
\end{verbatim}

\begin{Shaded}
\begin{Highlighting}[]
\NormalTok{naVec[!}\KeywordTok{is.na}\NormalTok{(naVec)] ##保留所有在is.na()檢查回傳FALSE的元素}
\end{Highlighting}
\end{Shaded}

\begin{verbatim}
## [1] "a" "b" "d" "e"
\end{verbatim}

若資料型態為資料框,可使用\texttt{complete.cases}來選出完整的資料列,如果資料列是完整的,則會回傳真TRUE

\begin{Shaded}
\begin{Highlighting}[]
\KeywordTok{head}\NormalTok{(airquality)}
\end{Highlighting}
\end{Shaded}

\begin{verbatim}
##   Ozone Solar.R Wind Temp Month Day
## 1    41     190  7.4   67     5   1
## 2    36     118  8.0   72     5   2
## 3    12     149 12.6   74     5   3
## 4    18     313 11.5   62     5   4
## 5    NA      NA 14.3   56     5   5
## 6    28      NA 14.9   66     5   6
\end{verbatim}

\begin{Shaded}
\begin{Highlighting}[]
\KeywordTok{complete.cases}\NormalTok{(airquality) }
\end{Highlighting}
\end{Shaded}

\begin{verbatim}
##   [1]  TRUE  TRUE  TRUE  TRUE FALSE FALSE  TRUE  TRUE  TRUE FALSE FALSE
##  [12]  TRUE  TRUE  TRUE  TRUE  TRUE  TRUE  TRUE  TRUE  TRUE  TRUE  TRUE
##  [23]  TRUE  TRUE FALSE FALSE FALSE  TRUE  TRUE  TRUE  TRUE FALSE FALSE
##  [34] FALSE FALSE FALSE FALSE  TRUE FALSE  TRUE  TRUE FALSE FALSE  TRUE
##  [45] FALSE FALSE  TRUE  TRUE  TRUE  TRUE  TRUE FALSE FALSE FALSE FALSE
##  [56] FALSE FALSE FALSE FALSE FALSE FALSE  TRUE  TRUE  TRUE FALSE  TRUE
##  [67]  TRUE  TRUE  TRUE  TRUE  TRUE FALSE  TRUE  TRUE FALSE  TRUE  TRUE
##  [78]  TRUE  TRUE  TRUE  TRUE  TRUE FALSE FALSE  TRUE  TRUE  TRUE  TRUE
##  [89]  TRUE  TRUE  TRUE  TRUE  TRUE  TRUE  TRUE FALSE FALSE FALSE  TRUE
## [100]  TRUE  TRUE FALSE FALSE  TRUE  TRUE  TRUE FALSE  TRUE  TRUE  TRUE
## [111]  TRUE  TRUE  TRUE  TRUE FALSE  TRUE  TRUE  TRUE FALSE  TRUE  TRUE
## [122]  TRUE  TRUE  TRUE  TRUE  TRUE  TRUE  TRUE  TRUE  TRUE  TRUE  TRUE
## [133]  TRUE  TRUE  TRUE  TRUE  TRUE  TRUE  TRUE  TRUE  TRUE  TRUE  TRUE
## [144]  TRUE  TRUE  TRUE  TRUE  TRUE  TRUE FALSE  TRUE  TRUE  TRUE
\end{verbatim}

\begin{Shaded}
\begin{Highlighting}[]
\KeywordTok{head}\NormalTok{(airquality[}\KeywordTok{complete.cases}\NormalTok{(airquality),]) ##保留所有在complete.cases()檢查回傳TRUE的元素}
\end{Highlighting}
\end{Shaded}

\begin{verbatim}
##   Ozone Solar.R Wind Temp Month Day
## 1    41     190  7.4   67     5   1
## 2    36     118  8.0   72     5   2
## 3    12     149 12.6   74     5   3
## 4    18     313 11.5   62     5   4
## 7    23     299  8.6   65     5   7
## 8    19      99 13.8   59     5   8
\end{verbatim}

利用演算法補值也是一種解決辦法,可參考\_skydome20\_的\href{http://www.rpubs.com/skydome20/R-Note10-Missing_Value}{R筆記--(10)遺漏值處理(Impute
Missing Value)}教學。

\section{綜合練習範例Case study}\label{manCase}

在本範例中,介紹使用\texttt{SportsAnalytics} \citep{R-SportsAnalytics}
package 撈取NBA各球員的數據,並加以觀察分析。

\subsection{載入資料}

首先用\texttt{library()}函數將\texttt{SportsAnalytics}套件載入
(若尚未安裝此套件者,必須先安裝套件,可參考Chapter
\ref{intro}),並利用套件內提供的\texttt{fetch\_NBAPlayerStatistics()}函數,將對應年份之資料取出。

\begin{Shaded}
\begin{Highlighting}[]
\KeywordTok{library}\NormalTok{(SportsAnalytics)}
\NormalTok{NBA1516<-}\KeywordTok{fetch_NBAPlayerStatistics}\NormalTok{(}\StringTok{"15-16"}\NormalTok{)}
\end{Highlighting}
\end{Shaded}

\subsection{資料總覽}

資料取出後,可用\texttt{str()}函數總覽\texttt{NBA1516}這個資料框的欄位與欄位類別

\begin{Shaded}
\begin{Highlighting}[]
\KeywordTok{str}\NormalTok{(NBA1516)}
\end{Highlighting}
\end{Shaded}

\begin{verbatim}
## 'data.frame':    476 obs. of  25 variables:
##  $ League             : Factor w/ 1 level "NBA": 1 1 1 1 1 1 1 1 1 1 ...
##  $ Name               : chr  "Quincy Acy" "Jordan Adams" "Steven Adams" "Arron Afflalo" ...
##  $ Team               : Factor w/ 31 levels "ATL","BOS","BRO",..: 27 15 22 20 19 13 28 26 12 15 ...
##  $ Position           : Factor w/ 5 levels "C","PF","PG",..: 4 5 1 5 1 1 2 2 2 5 ...
##  $ GamesPlayed        : int  59 2 80 71 59 60 74 9 79 64 ...
##  $ TotalMinutesPlayed : int  877 15 2019 2359 863 802 2260 37 1601 1622 ...
##  $ FieldGoalsMade     : int  119 2 261 354 150 134 536 5 191 215 ...
##  $ FieldGoalsAttempted: int  214 6 426 799 314 225 1045 10 370 469 ...
##  $ ThreesMade         : int  19 0 0 91 0 0 0 0 0 15 ...
##  $ ThreesAttempted    : int  49 1 0 238 1 0 16 0 0 42 ...
##  $ FreeThrowsMade     : int  50 3 114 110 52 60 259 0 46 90 ...
##  $ FreeThrowsAttempted: int  68 5 196 131 62 84 302 0 73 138 ...
##  $ OffensiveRebounds  : int  65 0 218 23 75 86 175 2 162 104 ...
##  $ TotalRebounds      : int  188 2 531 266 269 288 631 6 424 297 ...
##  $ Assists            : int  27 3 61 145 32 50 110 0 76 70 ...
##  $ Steals             : int  29 3 42 25 19 47 38 1 26 109 ...
##  $ Turnovers          : int  27 2 84 82 54 64 99 1 69 78 ...
##  $ Blocks             : int  24 0 89 10 36 68 81 2 42 18 ...
##  $ PersonalFouls      : int  103 2 223 142 134 139 151 1 147 175 ...
##  $ Disqualifications  : int  0 0 2 1 0 1 0 0 1 1 ...
##  $ TotalPoints        : int  307 7 636 909 352 328 1331 10 428 535 ...
##  $ Technicals         : int  3 0 2 1 2 0 0 0 0 1 ...
##  $ Ejections          : int  0 0 0 0 0 0 0 0 0 0 ...
##  $ FlagrantFouls      : int  0 0 0 0 0 0 0 0 0 0 ...
##  $ GamesStarted       : int  29 0 80 57 17 5 74 0 28 56 ...
\end{verbatim}

可以發現此\texttt{NBA1516}資料框內有476筆球員資料(觀察值,
obs),每筆資料有25個欄位 (variables)。 \#\#\#資料預覽
如果想看資料框內容,可用\texttt{head()}和\texttt{tail()}快速瀏覽部分資料

\begin{Shaded}
\begin{Highlighting}[]
\KeywordTok{head}\NormalTok{(NBA1516)}
\end{Highlighting}
\end{Shaded}

\begin{verbatim}
##   League          Name Team Position GamesPlayed TotalMinutesPlayed
## 1    NBA    Quincy Acy  SAC       SF          59                877
## 2    NBA  Jordan Adams  MEM       SG           2                 15
## 3    NBA  Steven Adams  OKL        C          80               2019
## 4    NBA Arron Afflalo  NYK       SG          71               2359
## 5    NBA Alexis Ajinca  NOR        C          59                863
## 6    NBA  Cole Aldrich  LAC        C          60                802
##   FieldGoalsMade FieldGoalsAttempted ThreesMade ThreesAttempted
## 1            119                 214         19              49
## 2              2                   6          0               1
## 3            261                 426          0               0
## 4            354                 799         91             238
## 5            150                 314          0               1
## 6            134                 225          0               0
##   FreeThrowsMade FreeThrowsAttempted OffensiveRebounds TotalRebounds
## 1             50                  68                65           188
## 2              3                   5                 0             2
## 3            114                 196               218           531
## 4            110                 131                23           266
## 5             52                  62                75           269
## 6             60                  84                86           288
##   Assists Steals Turnovers Blocks PersonalFouls Disqualifications
## 1      27     29        27     24           103                 0
## 2       3      3         2      0             2                 0
## 3      61     42        84     89           223                 2
## 4     145     25        82     10           142                 1
## 5      32     19        54     36           134                 0
## 6      50     47        64     68           139                 1
##   TotalPoints Technicals Ejections FlagrantFouls GamesStarted
## 1         307          3         0             0           29
## 2           7          0         0             0            0
## 3         636          2         0             0           80
## 4         909          1         0             0           57
## 5         352          2         0             0           17
## 6         328          0         0             0            5
\end{verbatim}

\subsection{資料排序後篩選}

觀察資料框的組成後,我們想要找出\textbf{出場數}最\textbf{高}的前五名選手的所有資料,此時可以利用\texttt{order()}函數先\textbf{由大到小}排序(\texttt{decreasing\ =\ T})後,再用\texttt{{[},{]}}取子集。

\begin{Shaded}
\begin{Highlighting}[]
\NormalTok{NBA1516Order<-NBA1516[}\KeywordTok{order}\NormalTok{(NBA1516$GamesPlayed,}\DataTypeTok{decreasing =} \NormalTok{T),]}
\NormalTok{NBA1516Order[}\DecValTok{1}\NormalTok{:}\DecValTok{5}\NormalTok{,] ##逗號前方放1~5,表示取1~5列;逗號後方空白,表示要取所有欄位}
\end{Highlighting}
\end{Shaded}

\begin{verbatim}
##     League            Name Team Position GamesPlayed TotalMinutesPlayed
## 11     NBA Al-farouq Aminu  POR       SF          82               2342
## 37     NBA     Will Barton  DEN       SG          82               2355
## 48     NBA Bismack Biyombo  TOR       PF          82               1810
## 62     NBA    Corey Brewer  HOU       SG          82               1670
## 118    NBA    Gorgui Dieng  MIN        C          82               2222
##     FieldGoalsMade FieldGoalsAttempted ThreesMade ThreesAttempted
## 11             299                 719        126             349
## 37             426                 984        112             324
## 48             156                 288          0               1
## 62             212                 552         61             225
## 118            308                 578          6              20
##     FreeThrowsMade FreeThrowsAttempted OffensiveRebounds TotalRebounds
## 11             115                 156                98           498
## 37             216                 268                60           477
## 48             142                 226               182           655
## 62             105                 140                42           199
## 118            205                 248               156           584
##     Assists Steals Turnovers Blocks PersonalFouls Disqualifications
## 11      138     72       120     53           171                 0
## 37      204     71       139     39           147                 0
## 48       29     19        71    133           225                 2
## 62      109     84        78     19           168                 1
## 118     143     94       140     96           219                 0
##     TotalPoints Technicals Ejections FlagrantFouls GamesStarted
## 11          839          3         0             0           82
## 37         1180          2         0             0            1
## 48          454          3         0             0           22
## 62          590          0         0             0           12
## 118         827          1         0             0           39
\end{verbatim}

如果我們想要出\textbf{出場分鐘數}最\textbf{高}的前十名選手的\textbf{名字},一樣可以用\texttt{order()}函數先\textbf{由大到小}排序(\texttt{decreasing\ =\ T})後,再用\texttt{{[},{]}}取子集。

\begin{Shaded}
\begin{Highlighting}[]
\NormalTok{NBA1516OrderM<-NBA1516[}\KeywordTok{order}\NormalTok{(NBA1516$TotalMinutesPlayed,}\DataTypeTok{decreasing =} \NormalTok{T),]}
\NormalTok{NBA1516OrderM[}\DecValTok{1}\NormalTok{:}\DecValTok{10}\NormalTok{,}\StringTok{"Name"}\NormalTok{] ##逗號前方取1~10列;逗號後方放"Name",表示取名稱為Name之欄位}
\end{Highlighting}
\end{Shaded}

\begin{verbatim}
##  [1] "James Harden"     "Gordon Hayward"   "Kemba Walker"    
##  [4] "Trevor Ariza"     "Khris Middleton"  "Kyle Lowry"      
##  [7] "Marcus Morris"    "Andrew Wiggins"   "Paul George"     
## [10] "Gi Antetokounmpo"
\end{verbatim}

\subsection{欄位值篩選}

除了排序取值外,也可用欄位條件搜尋,舉例來說,可以取出所有波士頓賽爾迪克隊的選手資料,使用\texttt{subset()}函數

\begin{Shaded}
\begin{Highlighting}[]
\KeywordTok{subset}\NormalTok{(NBA1516,Team==}\StringTok{"BOS"}\NormalTok{)}
\end{Highlighting}
\end{Shaded}

\begin{verbatim}
##     League            Name Team Position GamesPlayed TotalMinutesPlayed
## 60     NBA   Avery Bradley  BOS       PG          76               2536
## 89     NBA     Coty Clarke  BOS     <NA>           4                  8
## 102    NBA     Jae Crowder  BOS       SF          73               2310
## 213    NBA     R.j. Hunter  BOS       SG          36                319
## 228    NBA   Jonas Jerebko  BOS       PF          78               1178
## 229    NBA    Amir Johnson  BOS       PF          79               1798
## 300    NBA   Jordan Mickey  BOS       PF          16                 59
## 340    NBA    Kelly Olynyk  BOS        C          69               1396
## 382    NBA    Terry Rozier  BOS       PG          39                310
## 400    NBA    Marcus Smart  BOS       PG          61               1666
## 416    NBA Jared Sullinger  BOS       PF          81               1917
## 422    NBA   Isaiah Thomas  BOS       PG          82               2647
## 433    NBA     Evan Turner  BOS       SG          81               2270
## 471    NBA     James Young  BOS       SG          29                200
## 476    NBA    Tyler Zeller  BOS        C          60                714
##     FieldGoalsMade FieldGoalsAttempted ThreesMade ThreesAttempted
## 60             456                1018        147             406
## 89               2                   4          2               2
## 102            359                 812        122             363
## 213             36                  98         19              63
## 228            118                 286         43             108
## 229            250                 427         10              43
## 300              8                  22          0               0
## 340            253                 556         85             210
## 382             29                 106          6              27
## 400            184                 529         61             241
## 416            351                 807         29             104
## 422            591                1382        167             465
## 433            343                 753         20              83
## 471             11                  36          6              26
## 476            138                 290          0               0
##     FreeThrowsMade FreeThrowsAttempted OffensiveRebounds TotalRebounds
## 60              96                 123                48           220
## 89               0                   0                 0             1
## 102            196                 239                70           373
## 213              6                   7                 2            37
## 228             61                  78                77           288
## 229             69                 121               178           505
## 300              5                  10                 6            13
## 340             96                 128                72           281
## 382              8                  10                24            63
## 400            129                 166                76           255
## 416            103                 161               194           673
## 422            474                 544                46           243
## 433            148                 179                50           397
## 471              1                   4                 4            26
## 476             88                 108                62           178
##     Assists Steals Turnovers Blocks PersonalFouls Disqualifications
## 60      158    117       109     19           164                 2
## 89        0      0         1      0             0                 0
## 102     135    126        83     35           198                 4
## 213      13     14        11      4            29                 0
## 228      62     20        52     24           137                 2
## 229     137     52        94     83           214                 1
## 300       1      0         1     11             5                 0
## 340     105     52        74     33           163                 3
## 382      37      6        19      1            23                 0
## 400     186     91        80     18           183                 1
## 416     187     75       102     47           209                 2
## 422     509     91       220      9           167                 1
## 433     359     80       169     28           139                 0
## 471       9      6         5      1            17                 0
## 476      29     10        46     22            97                 1
##     TotalPoints Technicals Ejections FlagrantFouls GamesStarted
## 60         1155          0         0             0           72
## 89            6          0         0             0            0
## 102        1036          3         0             0           73
## 213          97          0         0             0            0
## 228         340          1         0             0            0
## 229         579          0         0             0           76
## 300          21          0         0             0            0
## 340         687          1         0             0            8
## 382          72          0         0             0            0
## 400         558          2         0             0           10
## 416         834          2         0             0           73
## 422        1823          9         0             0           79
## 433         854          2         0             0           12
## 471          29          0         0             0            0
## 476         364          0         0             0            3
\end{verbatim}

\subsection{字串條件搜尋後篩選}

當然也可以結合\textbf{字串搜尋}函數\texttt{grepl()},將所有名字裡有``James''的選手資料取出

\begin{Shaded}
\begin{Highlighting}[]
\NormalTok{NBA1516[}\KeywordTok{grepl}\NormalTok{(}\StringTok{"James"}\NormalTok{,NBA1516$Name),]}
\end{Highlighting}
\end{Shaded}

\begin{verbatim}
##     League           Name Team Position GamesPlayed TotalMinutesPlayed
## 15     NBA James Anderson  SAC       SG          51                721
## 132    NBA    James Ennis  NOR       SF          22                329
## 178    NBA   James Harden  HOU       SG          82               3121
## 222    NBA   Lebron James  CLE       SF          76               2710
## 231    NBA  James Johnson  TOR       PF          57                924
## 239    NBA    James Jones  CLE       SG          48                466
## 286    NBA   James Mcadoo  GSW       SG          41                265
## 471    NBA    James Young  BOS       SG          29                200
##     FieldGoalsMade FieldGoalsAttempted ThreesMade ThreesAttempted
## 15              67                 178         23              86
## 132             54                 113         26              58
## 178            710                1617        236             656
## 222            737                1416         87             282
## 231            114                 240         20              66
## 239             59                 143         41             104
## 286             45                  84          1               2
## 471             11                  36          6              26
##     FreeThrowsMade FreeThrowsAttempted OffensiveRebounds TotalRebounds
## 15              22                  29                13            86
## 132             25                  34                21            42
## 178            720                 837                63           502
## 222            359                 491               111           565
## 231             39                  68                28           126
## 239             21                  26                 8            50
## 286             26                  49                30            58
## 471              1                   4                 4            26
##     Assists Steals Turnovers Blocks PersonalFouls Disqualifications
## 15       41     21        42     14            54                 0
## 132      21     16        19      5            28                 1
## 178     612    138       374     51           229                 1
## 222     512    104       249     49           143                 0
## 231      67     29        54     33            84                 0
## 239      14     11        13     10            50                 0
## 286      17     10        16      8            39                 0
## 471       9      6         5      1            17                 0
##     TotalPoints Technicals Ejections FlagrantFouls GamesStarted
## 15          179          0         0             0           15
## 132         159          0         0             0            5
## 178        2376          2         0             0           82
## 222        1920          3         0             0           76
## 231         287          0         0             0           32
## 239         180          1         0             0            0
## 286         117          0         0             0            1
## 471          29          0         0             0            0
\end{verbatim}

\chapter{探索式資料分析}\label{eda}

\chapter{資料視覺化}\label{vis}

\chapter{互動式資料呈現}\label{InteractiveGraphics}

\chapter{資料探勘}\label{datamining}

\chapter{從小數據到大數據分析}\label{big}

\chapter{軟體安裝介紹}\label{install}

\chapter*{作者資訊}\label{author}
\addcontentsline{toc}{chapter}{作者資訊}

\chapter{Placeholder}\label{placeholder}

\bibliography{packages.bib,book.bib}


\end{document}
